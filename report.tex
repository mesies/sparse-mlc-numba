\documentclass[10pt,letterpaper]{article}
\usepackage[top=0.85in,left=2.75in,footskip=0.75in,marginparwidth=2in]{geometry}
\usepackage{amssymb}
% use Unicode characters - try changing the option if you run into troubles with special characters (e.g. umlauts)
\usepackage[utf8]{inputenc}

% clean citations
\usepackage{cite}
\usepackage[nottoc]{tocbibind}

% hyperref makes references clicky. use \url{www.example.com} or \href{www.example.com}{description} to add a clicky url
%\usepackage{nameref,hyperref}

% line numbers
\usepackage[right]{lineno}

% improves typesetting in LaTeX
\usepackage{microtype}
\DisableLigatures[f]{encoding = *, family = * }

% text layout - change as needed
\raggedright
\setlength{\parindent}{0.5cm}
\textwidth 5.25in 
\textheight 8.75in

% Remove % for double line spacing
%\usepackage{setspace} 
%\doublespacing

% use adjustwidth environment to exceed text width (see examples in text)
\usepackage{changepage}

% adjust caption style
\usepackage[aboveskip=1pt,labelfont=bf,labelsep=period,singlelinecheck=off]{caption}

% remove brackets from references
\makeatletter
\renewcommand{\@biblabel}[1]{\quad#1.}
\makeatother

% headrule, footrule and page numbers
\usepackage{lastpage,fancyhdr,graphicx}
\usepackage{epstopdf}
\pagestyle{myheadings}
\pagestyle{fancy}
\fancyhf{}
\rfoot{\thepage/\pageref{LastPage}}
\renewcommand{\footrule}{\hrule height 2pt \vspace{2mm}}
\fancyheadoffset[L]{2.25in}
\fancyfootoffset[L]{2.25in}

\usepackage{color}


\definecolor{Gray}{gray}{.25}


\usepackage{graphicx}

% use if you want to put caption to the side of the figure - see example in text
\usepackage{sidecap}

% use for have text wrap around figures
\usepackage{wrapfig}
\usepackage[pscoord]{eso-pic}
\usepackage[fulladjust]{marginnote}
\reversemarginpar

% document begins here
\begin{document}
\vspace*{0.35in}
	%TODO Change title
	% title goes here:
	\begin{flushleft}
		{\Large
			\textbf\newline{Implementation of Multi-Label Classification in Sparse Matrices}
		}
		\newline
		\\
		Napoleon Maraidonis\textsuperscript{1} under the supervision of
		Michalis Titsias\textsuperscript{1}		
						
		\bigskip
		\bf{1} Department of Informatics, Athens University of Economics and Business
		\\		
	\end{flushleft}
	
	\section*{Abstract}
	Fitting multiple figures into very tight manuscripts while keeping it pleasant to read is challenging. Therefore figures are often simply attached to the very end of a manuscript file. While easier for the authors, this practice is inconvenient for readers. This \LaTeX template shows how to generate a compiled PDF with figures embedded into the text. It provides several examples of how to embed figures or tables directly into the text thus giving you a range of options from which you should choose the one best suited for your manuscript. Check out Schlegel et al., (2016) as example of use \cite{MLC_madrid} also \cite{extreme_MLC}.
	
	% the * after section prevents numbering
	\section*{Introduction}
	
	Introduction \cite{extreme_MLC_omar} \cite{extreme_MLC_rep} \cite{MLC_finland}
	

	\section*{Notation}
	%	\begin{center}
		Lowercase bold letter denotes a n * k matrix where n is rows and k is columns   $\textbf{w} \in  {\Bbb R}^{n * k}$\\
		Lowercase bold letter with subscript denotes the n-nth row of the matrix  
		$\textbf{x}_{n} \in  {\Bbb R}^{dk}$\\
		Lowercase letter will denote a real number unless stated otherwise
		$x \in {\Bbb R}$\\
		Uppercase Italics letter denotes a set
		$\textit{T} $
	%	\end{center}

	\newpage
	
	
	\section*{Prototype Structure and Documentation}
	\subsection*{General Structure}
	The prototype consists of three main parts : (a) Multiclass Classification Algorithm, (b) Multy Label Classifiactation Interface (c) Score function.  More specifically :

	\subparagraph{a}Firstly, the implementation of any Multiclass Classification Algorithm will be required: multiclass logistic regression was chosen for simplicity. 
	The sparsity of the training data will be put into account while implementing the algorithm and stochastic gradient descent will be used for the optimization of the loss function.
	%TODO Another section for more specifics?
	
	\subparagraph{b}Classifier chains \cite{MLC_finland} will be used.
	%TODO Justify use of CC
	
	\subparagraph{c}Scoring will be made according to the formula:
	\begin{center}
		$accuracy \triangleq \frac{|T \cap P|}{|T \cup P |}$
	\end{center}
	%TODO Explain formula\ Citation Needed
	
	\subsubsection*{Notes}
	\begin{itemize}
		\item The implementation of any Multiclass Classification Algorithm will follow the following contract : The methods train(Xtrain,Ytrain), and predict(Xtest) will be implemented.
		\item The implementation of any Multy Label Classifiactation Interface will follow the following contract using the implemented Multiclass Classification Algorithm  : The methods train(Xtrain,Ytrain), and predict(Xtest) will be implemented.

	\end{itemize}
	
	\subsection*{Multiclass Logistic Regression Implementation}
	
	
	
	\newpage
	
	\section*{Results}

	\section*{Discussion}
	
	% You can use the \nameref{label} command to cite supporting items in the text.
	
\section*{Acknowledgments}

%This is where your bibliography is generated. Make sure that your .bib file is actually called library.bib
\bibliography{report}

%This defines the bibliographies style. Search online for a list of available styles.
\bibliographystyle{abbrv}
	
\end{document}